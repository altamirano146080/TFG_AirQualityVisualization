\chapter[Conclusions and Future Work]{Conclusions and Future Work}
\label{Chap5}

In this work, a web-based platform was developed to allow interactive exploration of atmospheric pollutant forecasts provided by the Copernicus Atmosphere Monitoring Service (CAMS). The main goal was to create a tool that runs entirely in a standard web browser, so users do not need to install specialized software or download large datasets. This approach makes it easy for students, researchers, and the general public to access and understand air quality forecasts directly from their devices.

\section{Conclusions}
The platform integrates an interactive map where users can visualize pollutant concentrations over Europe or the entire globe. By selecting a pollutant and a forecast hour, the map updates dynamically, showing the predicted distribution of the selected pollutant. Users can click on any point within the forecast domain to obtain detailed numerical values, which are automatically converted into standard units, making interpretation straightforward. The system also provides time-series charts for the clicked location, allowing users to see how pollutant concentrations evolve hour by hour. These charts can be exported for offline analysis, which is particularly useful for research purposes or educational exercises.

Animations play a key role in helping users understand how pollution spreads over time. By sequentially moving through forecast hours, the platform offers a visual sense of pollutant movement, although the performance can be affected by internet speed or device capabilities. Overall, the platform demonstrates that it is possible to make high-resolution atmospheric forecasts accessible and understandable, combining scientific rigor with simplicity and interactivity. Its open-source and browser-based design ensures that it can be easily maintained and adapted in the future, while providing a consistent and transparent way to interpret air quality data.

\section{Future Work}
Although the platform functions effectively as a prototype, several improvements could enhance usability and performance. One clear opportunity is to make the map occupy most of the screen, reducing empty space and allowing users to focus on the visualization. Controls, such as pollutant selectors or the time slider, could be implemented as floating or collapsible panels. This design would make the interface cleaner, allowing users to expand or hide controls as needed, without overwhelming the main view.

Another area for improvement concerns the interaction between regions and pollutants. Currently, when a user switches from one geographic domain to another, the selected pollutant may reset to a default option even if the same pollutant exists in the new region. Ideally, the platform should automatically retain the selected pollutant if it is available in the new region, only changing to a default option when the pollutant is not present. This enhancement would make the interface more intuitive and reduce unnecessary adjustments by the user, improving the overall experience when exploring forecasts across different geographic scales.

User interaction with the map could also be enhanced. Adding a visible marker at the point selected for historical data would help users immediately identify which location the time-series chart corresponds to. Furthermore, clarifying on-screen that clicking on the map retrieves historical information would make this feature more intuitive, particularly for users without prior experience with environmental data services.

Performance improvements could be achieved by preloading multiple forecast hours or caching data locally. Currently, when the forecast animation moves hour by hour, the map sometimes does not update smoothly due to network delays. By storing several layers in advance, the system could display them instantly, providing a more fluid experience. Similarly, allowing the simultaneous download of multiple forecast elements would reduce waiting times and improve responsiveness, particularly for larger geographic domains or slower connections.

Future developments could also expand the platform’s analytical capabilities. For example, users could overlay additional meteorological or environmental datasets to better understand factors influencing air quality. Customization of pollutant conversions or the addition of new variables would allow more detailed analyses, while optimizing the platform for mobile devices would make it accessible on a wider range of screens, from tablets to smartphones.

In summary, the platform provides an effective and intuitive way to visualize CAMS pollutant forecasts. While the current implementation already offers dynamic maps, point-specific queries, time-series visualization, and animations, there is significant potential for future enhancements. By improving usability, interactivity, and performance, the platform could serve as a comprehensive educational and research tool for understanding air quality and supporting informed decision-making regarding atmospheric pollution.


